% !TEX root = ../main.tex
\chapter{Open Quantum Systems} % Main chapter title
\label{chapter_open_quantum_systems} % Label for referencing this chapter

%------------------------------------------------------------------------------
%	SECTION 1: Introduction to Open Quantum Systems
%------------------------------------------------------------------------------

\section{Introduction to Open Quantum Systems}
\label{sec:introduction_open_quantum_systems}

Any real-world quantum system is not perfectly isolated. Instead, it interacts with its surrounding environment. This also happens in the most shielded experiments, at least to some degree. Unlike theoretical closed quantum systems that evolve unitarily according to the well known Schrödinger equation 

\begin{equation}
	i\hbar \frac{\partial}{\partial t} |\psi(t)\rangle = H(t) |\psi(t)\rangle ,
	\label{eq:SchrödingerEquation}
\end{equation}

\noindent
open quantum systems experience non-unitary evolution due to their coupling with an external environment or reservoir.
Whenever the system is driven by external perturbations, this coupling makes the system relax back to the equilibrium defined by this environment ~\cite{breuerpetruccione2009theoryopenquantum, weiss2012quantumdissipativesystems}.
Open quantum systems theory emerges naturally from this realization, that perfect isolation of a quantum systems is practically impossible. For instance, atoms are subject to electromagnetic field fluctuations (vacuum fluctuations)~\cite{breuerpetruccione2009theoryopenquantum}, quantum dots in solid-state environments couple to phonon baths~\cite{weiss2012quantumdissipativesystems}, and molecular systems interact with surrounding solvent molecules~\cite{mukamel1995principlesnonlinearoptical}. Quantum computers are particularly sensitive to environmental noise, which can rapidly destroy quantum coherence~\cite{laddetal2010quantumcomputers}, while biological quantum systems such as photosynthetic complexes operate in inherently noisy cellular environments. \todoref{schlosshauer2007decoherencebook}. 
These diverse scenarios all require a theoretical framework that accounts for the coexistence of quantum effects and environmental influences.

\subsection{Central concepts}
The interaction with the environment leads to several fundamental phenomena that distinguish open systems from isolated, closed quantum systems.

\noindent
A central effect is \textbf{decoherence}, where quantum superposition states lose their phase relationships due to entanglement with the environment (\textbf{dephasing}). As a result, pure quantum states evolve into classical statistical mixtures, and the system's ability to exhibit quantum interference is diminished. The characteristic time scale for this process, known as the \emph{coherence time}, is of utmost importance in experiments, especially in high-resolution spectroscopy, quantum information processing, and quantum optics.

\noindent
In addition to dephasing, the system-environment interaction leads to \textbf{dissipation} or \textbf{thermalization}, where energy is exchanged between the system and its surroundings, causing the system to relax toward a thermal state determined by the environment's temperature.

\noindent
Understanding and controlling these effects is crucial for the design of quantum technologies, such as quantum computers and sensors, where environmental noise can rapidly destroy fragile quantum states~\cite{laddetal2010quantumcomputers}.  \todoref{find better ref schlosshauer2007decoherencebook}

\noindent
In summary, open quantum systems theory provides the framework to describe how quantum systems lose their "quantumness" and transition toward classical behavior due to unavoidable interactions with their environment.


\subsubsection{Overview of Main Approaches}
\label{subsec:overview_main_approaches_oqs}

\noindent
A variety of theoretical frameworks have been developed to describe the dynamics of open quantum systems, each with its own range of validity and underlying assumptions. Some are presented here:

\noindent
In \textbf{Markovian dynamics} it is assumed that the environment has no memory: the future of the system depends only on its present state. This is valid when the environmental correlation time is much shorter than the system's characteristic timescale \todoref{find reference}. 

\todoidea{Maybe not needed:} 
\noindent
In contrast, in \textbf{non-Markovian dynamics} the memory effects of the environment are taken into account. The environment retains information about the system's past, leading to feedback and more complex evolution~\cite{breuerpetruccione2009theoryopenquantum, rivasetal2014quantumnonmarkovianitycharacterization}.
\noindent
In Time-convolutionless and Nakajima–Zwanzig master equations memory kernels describe such effects~\cite{breuerpetruccione2009theoryopenquantum, rivasetal2014quantumnonmarkovianitycharacterization}. The hierarchical equations of motion (HEOM) provide a numerically exact framework for strong coupling and non-Markovian, condensed-phase environments like liquids or solids~\cite{tanimura2020numericallyexactapproach}. This is the most computationally expensive method. Also Path-integral methods can be noted. They are based on the Feynman–Vernon influence functional and offer a non-perturbative route to non-Markovian dynamics~\cite{weiss2012quantumdissipativesystems}.


\paragraph{Stochastic Approaches}

\noindent
Stochastic Schrödinger equations and quantum trajectories unravel master equations into individual pure state evolutions ~\cite{vogtetal2013stochasticblochredfieldtheory, breuerpetruccione2009theoryopenquantum, carmichael1993opensystemsapproach}. These methods are particularly useful for simulations, as the desired precision decides the number of trajectories needed.

\paragraph{Master Equation Approaches}

\noindent
The Lindblad master equation gives the most general form for completely positive, trace-preserving dynamics under the Born-Markov approximation and is widely used in quantum optics and quantum information~\cite{breuerpetruccione2009theoryopenquantum, lindblad1976generatorsquantumdynamical}.
This thesis is focussed on using a more general master equation—the Redfield equation. It will be derived in the next section \ref{sec:Derivation_redfield_eq}.


\subsection{The Redfield Equation: A Central Tool}

Originally developed by A.G. Redfield in 1957 and 1965 for nuclear magnetic resonance relaxation phenomena~\cite{redfield1965theoryrelaxationprocesses}, the Redfield equation describes the time evolution of the reduced density matrix of a quantum system weakly coupled to a thermal environment.

\noindent
Unlike phenomenological approaches that introduce dissipation and dephasing \emph{ad hoc}, the Redfield equation relates them to environmental correlation functions or spectral densities ~\cite{breuerpetruccione2009theoryopenquantum, weiss2012quantumdissipativesystems}.

\noindent
The following requirements must be fulfilled by the final derived Redfield equation:

\begin{enumerate}
	\item The equation should be linear in the system density matrix $\dot{\rho}_S(t) = F(\rho_S(t))$ (reduced equation of motion).
	\item The equation should be Markovian, meaning that the evolution of the system density matrix at time $t$ only depends on the state of the system at time $t$ and not on its past history.
	\item The equation should be trace-preserving, meaning that $\mathrm{Tr}[\rho_S(t)] = \mathrm{Tr}[\rho_S(0)]$ for all times $t$.
\end{enumerate}

It is important to note that the Redfield equation, while preserving trace and Hermiticity, does not guarantee complete positivity of the density matrix—a fundamental requirement for physical quantum states~\cite{rivasetal2010markovianmasterequations}. 
So care must be taken when determining when the Redfield equation is useful ~\cite{redfield1965theoryrelaxationprocesses, rivasetal2014quantumnonmarkovianitycharacterization, lietal2018conceptsquantumnonmarkovianity}.
\todoidea{explain why we choose the Redfield equation for this thesis, despite the limitations ->  it represents the best compromise between accuracy and computational efficiency for our systems of interest. -> Mostly used in biological systems and quantum chemistry}


\noindent
In this section, we will derive the Redfield equation starting from the fundamental microscopic dynamics of the system-environment composite and demonstrate how it emerges as an effective description for the reduced system dynamics. We will then examine the environmental correlation functions and spectral densities that characterize the bath properties and determine the system's relaxation behavior.

%------------------------------------------------------------------------------
%	SECTION 2: Derivation of the Redfield Equation
%------------------------------------------------------------------------------

\section{Derivation from Microscopic Dynamics}
\label{sec:Derivation_redfield_eq}

\noindent
This derivation follows the approach presented in \cite{manzano2020shortintroductionlindblad} and can also be found in standard textbooks such as Breuer and Petruccione \cite{breuerpetruccione2009theoryopenquantum}. \todoidea{However here we will be more detailed and explicit in the steps.}

\subsection{Mathematical Preliminaries}
\label{subsec:preliminaries_tools}

First we need a few mathematical tools and definitions that will be used in the derivation.
\noindent
\paragraph{Partial trace.}

\noindent
The partial trace is a way to simplify a quantum system's description by removing certain parts through averaging. It's like the reverse of combining spaces with a tensor product. This is handy when only one part of a combined system is of interest \cite{lambertetal2024qutip5quantum}.
For a bipartite Hilbert space $\mathcal{H_A} \otimes \mathcal{H_B}$, the partial trace over $\mathcal{H_B}$ is the unique linear map $\mathrm{Tr}_B: \mathcal{B}_1(\mathcal{H_A} \otimes \mathcal{H_B}) \to \mathcal{B}_1(\mathcal{H_A})$ satisfying $\mathrm{Tr}[(\mathrm{Tr}_B X) A] = \mathrm{Tr}[X (A \otimes I_B)]$ for all $A \in \mathcal{B}(\mathcal{H_A})$, where  $\mathcal{B}(\mathcal{H})$ is the algebra of bounded operators on $\mathcal{H}$.
and $\mathcal{B}_1$ denotes trace-class operators, namely those $X$ with finite trace norm 
\[ 
  \| X \|_1 = \operatorname{Tr} \sqrt{X^\dagger X} 
\]

\noindent
This ensures expectation values of local observables $A \otimes I_B$ are preserved. In an orthonormal basis $\{|b_i\rangle\}$ of $\mathcal{H_B}$, it acts as \cite{steebhardy2018problemssolutionsquantum}:

\begin{equation} \label{eq:ho_partial_trace_definition}
	\mathrm{Tr}_B X = \sum_i (I_A \otimes \langle b_i|) \, X \, (I_A \otimes |b_i\rangle).
\end{equation}

\noindent
With this definition thermal expectation values take the form

\begin{equation} \label{eq:ho_expectation_value} \langle A \rangle = \mathrm{Tr}[\rho A] = \frac{1}{Z} \sum_n e^{-\beta E_n} A_{nn}
\end{equation}

\begin{equation} \label{eq:ho_beta_definition}
	\beta = \frac{1}{k_{\mathrm{B}} T}
\end{equation}


\subsection{Setup: System + Environment}

\noindent
We consider a quantum system of interest interacting with an environment, which has infinite degrees of freedom. The total Hilbert space $\mathcal{H}_T$ is the tensor product of the Hilbert of the system $\mathcal{H}_S$ and the one of the environment $\mathcal{H}_E$:

\begin{equation}
	\mathcal{H}_T = \mathcal{H}_S \otimes \mathcal{H}_E.
	\label{eq:Total_Hilbert_Space}
\end{equation}

\noindent
The evolution of the total system is governed by the Liouville–von Neumann equation:

\begin{equation}
	\dot{\rho}_T(t) = -i[H_T, \rho_T(t)],
	\label{eq:Von_Neumann_Equation}
\end{equation}

\noindent
where $\rho_T(t)$ is the density matrix of the total system, $H_T$ is the total Hamiltonian, and we use units where $\hbar = 1$. This is the correspondence of the Schrödinger equation Eq.\eqref{eq:SchrödingerEquation} for more general classically mixed states 

\begin{equation}
	\rho = \sum_i p_i |\psi_i\rangle \langle \psi_i|, \quad p_i \geq 0, \quad \sum_i p_i = 1.
\end{equation}

\noindent
Without loss of generality, the total Hamiltonian $H_T \in \mathcal{B}(\mathcal{H}_T)$ can be decomposed as:

\begin{equation}
	H_T = H_S \otimes \mathds{1}_E + \mathds{1}_S \otimes H_E + \alpha H_I,
	\label{eq:Total_Hamiltonian}
\end{equation}

\noindent
where $H_S$ and $H_E$ act on the individual Hilbert spaces $\mathcal{H}_S$ and $\mathcal{H}_E$, respectively, while $H_I$ acts on the composite space $\mathcal{H}_T$ (system--environment interaction). The dimensionless parameter $\alpha$ is the coupling strength parameter.

\noindent The interaction Hamiltonian is typically written in the form:

\begin{equation}
	H_I = \sum_i S_i \otimes E_i,
	\label{eq:Interaction_Hamiltonian}
\end{equation}

\noindent
where $S_i$ are system operators and $E_i$ are environment operators. These will be concretized later in Sec.~\ref{sec:harmonic_oscillator_baths}.

\subsection{Interaction Picture}

\noindent
To describe the system dynamics, we move to the interaction picture where the operators evolve with respect to $H_S + H_E$. Any arbitrary operator $O$ in the Schrödinger picture takes the form:

\begin{equation}
	\hat{O}(t) = e^{i(H_S+H_E)t} O e^{-i(H_S+H_E)t},
	\label{eq:Interaction_Picture_Operators}
\end{equation}

\noindent
in the interaction picture. States now evolve only according to the interaction Hamiltonian $H_I$, and the Liouville-von Neumann equation becomes:

\begin{equation}
	\dot{\hat{\rho}}_T(t) = -i \alpha [\hat{H}_I(t), \hat{\rho}_T(t)],
	\label{eq:LiouvilleVN}
\end{equation}

\noindent
which can be formally integrated as:

\begin{equation}
	\hat{\rho}_T(t) = \hat{\rho}_T(0) - i \alpha \int_0^t ds [\hat{H}_I(s), \hat{\rho}_T(s)].
	\label{eq:Formal_Integration}
\end{equation}

\noindent
Inserting this back into Eq.~\eqref{eq:LiouvilleVN} yields:

\begin{equation}
	\dot{\hat{\rho}}_T(t) = -i \alpha \left[ \hat{H}_I(t), \hat{\rho}_T(0) \right]
	- \alpha^2 \int_0^t \left[ \hat{H}_I(t), \left[ \hat{H}_I(s), \hat{\rho}_T(s) \right] \right] ds,
	\label{eq:Second_Order_Expansion}
\end{equation}

\noindent
The iteration can be repeated, leading to a series expansion in powers of $\alpha$:

\begin{equation}
	\dot{\hat{\rho}}_T(t) = -i \alpha \left[ \hat{H}_I(t), \hat{\rho}_T(0) \right]
	- \alpha^2 \int_0^t \left[ \hat{H}_I(t), \left[ \hat{H}_I(s), \hat{\rho}_T(s) \right] \right] ds + \mathcal{O} (\alpha^3).
	\label{eq:Second_Order_Expansion_truncated}
\end{equation}

\noindent
We truncate at second order, justified by the weak coupling assumption ($\alpha \ll 1$), which represents the \textbf{Born approximation}.

\noindent
Equation~\eqref{eq:Second_Order_Expansion_truncated} still contains the full history through $\hat{\rho}_T(s)$ inside the integral and is therefore \emph{non-Markovian}. We now assume that $\hat{\rho}_T$ is approximately constant on the bath correlation time by replacing $\hat{\rho}_T(s) \to \hat{\rho}_T(t)$ inside the integrand, which represents a time-convolutionless approximation. Doing so yields

\begin{equation}
	\dot{\hat{\rho}}_T(t) = -i \alpha \big[ \hat{H}_I(t), \hat{\rho}_T(0) \big]
	- \alpha^2 \int_0^t ds\, \big[ \hat{H}_I(t), [ \hat{H}_I(s), \hat{\rho}_T(t)] \big],
	\label{eq:Second_Order_Expansion_wo_third}
\end{equation}

\noindent
which is now local in $\hat{\rho}_T(t)$ (but still retains an explicit upper integration limit $t$ so is not yet Markovian). This step will be further justified once we restrict attention to the \emph{reduced} dynamics and invoke separation of time scales.

\noindent
The exact equation of motion for $\rho(t)$ involves the full many-body dynamics of the environment, which is generally intractable. Thats why the next step is to trace out the environmental degrees of freedom.


\subsection{Reduced Dynamics and Markovian Approximation}

\noindent
Since we are interested in the dynamics of the system alone, we define the reduced density matrix using the partial trace operation defined in Eq.~\eqref{eq:ho_partial_trace_definition}:

\begin{equation}
	\rho_S(t)= \mathrm{Tr}_E[\rho_T(t)].
	\label{eq:Reduced_Density_Matrix}
\end{equation}

Thus Eq.~\eqref{eq:Second_Order_Expansion_wo_third} becomes
\begin{equation}
	\dot{\hat{\rho}}_S(t) = -i \, \alpha \, \mathrm{Tr}_E\big[\,\hat{H}_I(t),\hat{\rho}_T(0)\,\big]
	- \alpha^2 \int_0^t ds\, \mathrm{Tr}_E \big[\,\hat{H}_I(t), [\hat{H}_I(s), \hat{\rho}_T(t)] \,\big] \notag
	\label{eq:Reduced_Density_Matrix_Evolution}
\end{equation}

\paragraph{Eliminating the first-order contribution.}

\noindent
The first (order-$\alpha$) term in Eq.~\eqref{eq:Reduced_Density_Matrix_Evolution} will vanish after tracing over the environment if the interaction operators satisfy 
\begin{equation}
	\langle E_i \rangle_0 \equiv \mathrm{Tr}_E[E_i \, \hat{\rho}_E(0)] = 0,
	\label{eq:Zero_Mean_Condition}
\end{equation}

\noindent
This is achieved by assuming, that the total system starts in a \textbf{separable} product state of the form:

\begin{equation}
	\hat{\rho}_T(0) = \hat{\rho}_S(0) \otimes \hat{\rho}_E(0),
	\label{eq:Initial_Product_State}
\end{equation}

\noindent
Some textbooks impose the condition \eqref{eq:Zero_Mean_Condition} as a starting assumption, while we go a step further. When it is not initially the case one can \emph{always} enforce it by redefining the Hamiltonian through a shift:
\begin{equation}
	H_T = H_S' + H_E + \alpha H_I',
	\label{eq:Shifted_Total_Hamiltonian}
\end{equation}

\noindent
with

\begin{equation}
	H_I' = \sum_i S_i \otimes E_i', \qquad E_i' = E_i - \langle E_i \rangle_0, \qquad \langle E_i \rangle_0 \equiv \mathrm{Tr}_E[E_i \hat{\rho}_E(0)],
	\label{eq:Shifted_Interaction_Hamiltonian}
\end{equation}

\noindent
and a correspondingly shifted system Hamiltonian

\begin{equation}
	H_S' = H_S + \alpha \sum_i S_i \langle E_i \rangle_0.
	\label{eq:Shifted_System_Hamiltonian}
\end{equation}

\noindent
This ``renormalization'' merely redefines system energy levels and does not affect dissipative structure; hence, without loss of generality we assume the shift performed so that $\langle E_i \rangle_0 = 0$ and the order-$\alpha$ term can be discarded after tracing.

\noindent
Using the shifted form Eqs.~\eqref{eq:Shifted_Interaction_Hamiltonian}--\eqref{eq:Shifted_System_Hamiltonian} (so that $\langle E_i \rangle_0 = 0$), the first-order term in Eq.~\eqref{eq:Reduced_Density_Matrix_Evolution} vanishes. Explicitly, for the order-$\alpha$ contribution one has
\begin{align}
	\sum_i \mathrm{Tr}_E\big[ S_i \otimes E_i, \hat{\rho}_S(0) \otimes \hat{\rho}_E(0)\big]
	 & = \sum_i \big(S_i \hat{\rho}_S(0) - \hat{\rho}_S(0) S_i\big) \, \mathrm{Tr}_E[E_i \hat{\rho}_E(0)] = 0,
	\label{eq:Trace_Relation_first_part}
\end{align}

\noindent
which implements, that a static bath mean field can be absorbed into $H_S$.

\noindent
The reduced equation of motion becomes:

\begin{align}
	\dot{\hat{\rho}}_S(t) & = -i \, \alpha \, \mathrm{Tr}_E\big[\,\hat{H}_I(t),\hat{\rho}_T(0)\,\big]
	- \alpha^2 \int_0^t ds\, \mathrm{Tr}_E \big[\,\hat{H}_I(t), [\hat{H}_I(s), \hat{\rho}_T(t)] \,\big] \notag                     \\
	                      & = - \alpha^2 \int_0^t ds\, \mathrm{Tr}_E \big[\,\hat{H}_I(t), [\hat{H}_I(s), \hat{\rho}_T(t)] \,\big].
	\label{eq:Partial_Trace_Derivation}
\end{align}

\noindent
To make the time-translation structure explicit, set $s' = t-s$ (equivalently, $ds' = -ds$). When $s$ runs from $0$ to $t$, $s'$ runs from $t$ to $0$. Reversing the limits removes the minus sign, so the integration range remains $[0,t]$. Using this change and expanding the double commutator,
\begin{equation}
	\label{eq:double_comm_expansion_rule}
	\big[ A, [B, X] \big] = A B X - A X B - B X A + X B A,
\end{equation}

\noindent
we can rewrite Eq.~\eqref{eq:Partial_Trace_Derivation} in the form

\begin{align}
	\dot{\rho}_S(t) & = \alpha^2 \int_0^t ds \, \mathrm{Tr}_E \bigg\{
	\hat{H}_I(t) \big[ \hat{H}_I(t-s) \hat{\rho}_T(t) - \hat{\rho}_T(t) \, \hat{H}_I(t-s) \big] \notag                                   \\
	                & \qquad\qquad\qquad\; - \big[ \hat{H}_I(t-s) \hat{\rho}_T(t) - \hat{\rho}_T(t) \, \hat{H}_I(t-s) \big] \hat{H}_I(t)
	\bigg\}.
	\label{eq:Second_Order_Final_Expression}
\end{align}

\noindent
We now have to tighten the assumption of Eq. \eqref{eq:Initial_Product_State} and force, that throughout the evolution the total state remains close to a factorized form $\hat{\rho}_T(t) \approx \hat{\rho}_S(t) \otimes \hat{\rho}_E(t)$. This can again be stated as the \textbf{Born approximation} of weak coupling.

\noindent
By collecting all system operators and all environment operators, and inserting the interaction Hamiltonian Eq.~\eqref{eq:Interaction_Hamiltonian} explicitly, tracking the operators at time $t - s$ with $i'$ and at time $t$ with $i$, we obtain:

\begin{align}
	\dot{\hat{\rho}}_S(t) & = \alpha^2  \sum_{i, i'} \int_0^t ds
	\bigg\{
	\mathrm{Tr}_E \big[ \hat{S}_i(t) \hat{S}_{i'}(t-s) \hat{\rho}_S(t)      \otimes   \hat{E}_{i}(t) \hat{E}_{i'}(t-s) \hat{\rho}_E(t)  \big] -  \notag                         \\
	                      & \mathrm{Tr}_E \big[ \hat{S}_i(t) \hat{\rho}_S(t) \hat{S}_{i'}(t-s)      \otimes   \hat{E}_{i}(t) \hat{\rho}_E(t) \hat{E}_{i'}(t-s)  \big] - \notag \\
	                      & \mathrm{Tr}_E \big[ \hat{S}_{i'}(t-s) \hat{\rho}_S(t) \hat{S}_i(t)      \otimes   \hat{E}_{i'}(t-s) \hat{\rho}_E(t) \hat{E}_{i}(t)  \big] +  \notag \\
	                      & \mathrm{Tr}_E \big[ \hat{\rho}_S(t) \hat{S}_{i'}(t-s) \hat{S}_i(t)      \otimes   \hat{\rho}_E(t) \hat{E}_{i'}(t-s) \hat{E}_{i}(t)  \big]
	\bigg\}.
	\label{eq:Interaction_Hamiltonian_Expansion}
\end{align}

\noindent
Since the trace only acts on the environment, the system operators can be taken out of the trace, and we define the two point correlation functions with Eq.~\eqref{eq:ho_expectation_value}:

\begin{equation}
	C_{ii'}(t, s) \equiv \langle \hat{E}_{i}(t) \hat{E}_{i'}(t-s) \rangle = \mathrm{Tr}_E \big[\hat{E}_{i}(t) \hat{E}_{i'}(t-s) \hat{\rho}_E(t)\big],
	\label{eq:Environment_Correlation_Function}
\end{equation}

\noindent
The two point means that we "measure" the correlation operators $\hat{E}_{i}$ and $\hat{E}_{i}$ at two different times.
If the bath operators are Hermitian, i.e. $\hat{E}_i^\dagger = \hat{E}_i$,
and since the bath density matrix is Hermitian $ \hat{\rho}_E(t) = \hat{\rho}_E^\dagger(t)$, we can use the cyclic property of the trace to show that

\begin{equation}
	C^{*}_{ii'}(t, s) = \tilde{C}_{ii'}(t, s) \equiv \langle \hat{E}_{i'}(t-s) \hat{E}_{i}(t) \rangle = \mathrm{Tr}_E \big[\hat{E}_{i'}(t-s) \hat{E}_{i}(t) \hat{\rho}_E(t)\big],
	\label{eq:Environment_Correlation_Function_Conjugate}
\end{equation}

\noindent
and thus finally obtain the desired form of the Redfield equation

\begin{align}
	\dot{\hat{\rho}}_S(t) = \alpha^2  \sum_{i, i'} \int_0^t ds
	\bigg\{
	C_{ii'}(t, s) \big[ \hat{S}_i(t),  \hat{S}_{i'}(t-s) \hat{\rho}_S(t) \big] + \text{H.c.}
	\bigg\}.
	\label{eq:Redfield_Equation_Non_Markovian}
\end{align}

\noindent
For additionally having a stationary bath, $[H_E, \rho_E(0)]=0$, so in the interaction picture $\hat{\rho}_E(t)=\hat{\rho}_E(0)=\rho_E(0)$ the correlators depend only on the time difference $\tau= t - s$ since one operator, e.g. $\hat{E}_{i}$ can always be written as $\hat{E}_{i}(0) = e^{-i H_E t} \hat{E}_i(t) e^{+i H_E t}$. So here we have the useful condition:
\begin{equation}
	C^{*}_{ii'}(-\tau) = C_{ii'}(\tau).
\end{equation}

\vspace{1em}
\noindent
However the Eq. \eqref{eq:Redfield_Equation_Non_Markovian} is still not Markovian, since it carries an explicit integration over time $t$.
Now another approximation has to be made. Alongside the Born approximation, where correlations between system and environment are small, we also assume that any correlation of the environment decay on a timescale $\tau_E$ much shorter than a characteristic system evolution time $\tau_S$. The reduced density matrix $\rho_S(t)$ changes only on a much longer times $\tau_E \ll \tau_S$. Correlation functions $C_{ii'}(\tau)$ decay to negligible values for $\tau \gtrsim \tau_E$. Thus $ C(\tau) \approx 0$ for $\tau \gg 1$. Within the integral we may (i) replace the system operators $\hat{S}_i'(t-s)$ by its free interaction-picture evolution and (ii) extend the upper limit to infinity:

\begin{equation}
	\int_0^t ds\, C_{ii'}(t-s) f(s) \; \longrightarrow \; \int_0^{\infty} d\tau\, C_{ii'}(\tau) f(t), \qquad (t \gg \tau_E),
	\label{eq:Markov_extension_rule}
\end{equation}

\noindent
where we used the slow variation of $\rho_S(t)$ and the system operators \textit{only} on $\tau_E$. 
Applying this rule to Eq.~\eqref{eq:Partial_Trace_Derivation} yields the \emph{time-homogeneous} (Markovian) Redfield generator:

\begin{equation}
	\boxed{
		\dot{\hat{\rho}}_S(t) = - \alpha^2 \sum_{i,i'} \int_0^{\infty} d\tau \, \Big( C_{ii'}(\tau) [\hat{S}_i(t), \hat{S}_{i'}(t-\tau) \hat{\rho}_S(t)] + \text{H.c.}\Big).
	}
	\label{eq:Redfield_Markov_TimeLocal}
\end{equation}

\noindent
This step encapsulates the loss of memory: the generator now includes a very short time window $\tau_E$ where the system operators can be approximated as constant (frozen at time $\rho_S(t)$).


\paragraph{Eigenoperator (frequency) decomposition.}

\noindent
To proceed analytically and to connect with relaxation pathways, we decompose the system coupling operators into eigenoperators of the system Hamiltonian. Let $H_S = \sum_{\epsilon} \epsilon \, \Pi_{\epsilon}$ be the spectral resolution with projectors $\Pi_{\epsilon}$. Define the Bohr frequencies $\omega = \epsilon' - \epsilon$ and the corresponding interaction operators in the eigenbasis

\begin{equation}
	S_i(\omega) = \sum_{\epsilon' - \epsilon = \omega} \Pi_{\epsilon} S_i \Pi_{\epsilon'}.
	\label{eq:Eigenoperator_Decomposition}
\end{equation}

\noindent
In the interaction picture these acquire simple oscillatory phases:

\begin{equation}
	\hat{S}_i(t) = \sum_{\omega} e^{-i \omega t} S_i(\omega), \qquad \hat{S}_i(t-\tau) = \sum_{\omega'} e^{-i \omega'(t-\tau)} S_i(\omega').
	\label{eq:Interaction_Picture_Eigenoperators}
\end{equation}

\noindent
Inserting Eq.~\eqref{eq:Interaction_Picture_Eigenoperators} into Eq.~\eqref{eq:Redfield_Markov_TimeLocal} produces sums over oscillatory factors $e^{-i(\omega - \omega') t}$ multiplying integrals of $C_{ii'}(\tau) e^{i \omega' \tau}$. The inner commutator in Eq.~\eqref{eq:Redfield_Markov_TimeLocal} becomes
\begin{equation}
[\hat{S}_i(t),\, \hat{S}_{i'}(t-\tau)\rho]
= \sum_{\omega,\omega'} e^{-i(\omega - \omega')t} e^{+i \omega' \tau}
\big[ S_i(\omega),\, S_{i'}(\omega') \rho \big],
\tag{3}
\end{equation}
and the total Redfield equation reads
\begin{equation}
	\dot{\hat{\rho}}_S(t) = -\alpha^2 \sum_{i,i'} \sum_{\omega,\omega'} e^{-i(\omega - \omega')t}
	\left( \int_0^{\infty} d\tau\, C_{ii'}(\tau)\, e^{+i \omega' \tau} \right)
	\big[ S_i(\omega),\, S_{i'}(\omega') \hat{\rho}_S(t) \big]
	+ \text{H.c.}
	\label{eq:Redfield_Frequency_Decomposed}
\end{equation}

\noindent
Now we express the correlator in terms of the noise-power spectrum of the environment \cite{lambertetal2024qutip5quantum}.
\begin{equation}
	\tilde{S}(\omega) \equiv \int_{-\infty}^{\infty} d\tau\, C(\tau)\, e^{+i \omega \tau},
	\label{eq:Noise_Power_Spectrum}
\end{equation}
not to be confused with the system operators $S_i$.


\noindent
The exact one-sided Fourier (Laplace-) transform of the bath correlation function can be split into the power spectrum and an energy shift $ \lambda(\omega)$ as

\begin{align}
	\chi_{ii'}(\omega) &\equiv \int_0^{\infty} d\tau\, C_{ii'}(\tau)\, e^{+i \omega \tau} \\
					   &= \tfrac{1}{2}\tilde{S}_{ii'}(\omega)+i\,\lambda_{ii'}(\omega), \\
\end{align}

\noindent
with 
\begin{equation}
	\tilde{S}_{ii'}(\omega)=\chi_{ii'}(\omega)+\chi_{i'i}^*(\omega).
	\label{eq:Redfield_Rates_Definition}
\end{equation}

\noindent
which reduce Eq. \eqref{eq:Redfield_Frequency_Decomposed} to the compact form

\begin{equation}
	\boxed{
	\dot{\hat{\rho}}_S(t)
	= -\alpha^2 \sum_{i,i'} \sum_{\omega,\omega'} e^{-i(\omega - \omega')t}
	\, \chi_{ii'}(\omega') \,
	\big[ S_i(\omega),\, S_{i'}(\omega') \hat{\rho}_S(t) \big]
	+ \text{H.c.}
	}
\end{equation}

Going back to the Schrödinger picture, this yields
\begin{equation}
\dot{\rho}_S(t)=-\frac{i}{\hbar}[H_S+H_{\rm LS},\rho_S(t)]+\mathcal{D}_\mathrm{R}[\rho_S(t)],
\end{equation}
with the Lamb-shift Hamiltonian
\begin{equation}
	H_{\rm LS}=\frac{1}{\hbar^{2}}\sum_{\omega}\sum_{i, i'}\lambda_{ii'}(\omega)\,
	S_i^\dagger(\omega)S_{i'}(\omega),
	\label{eq:Lamb_Shift_Hamiltonian}
\end{equation}
which will directly be ignored in the following, and the Redfield dissipator
\begin{equation}
	\mathcal{D}_\mathrm{R}[\rho]
	=\frac{1}{2\hbar^{2}}\sum_{\omega,\omega'}\sum_{i, i'}
	e^{-i(\omega-\omega')t} \tilde{S}_{ii'}(\omega')\,
	\Big(
		S_{i'}(\omega')\rho S_i^\dagger(\omega)
		- S_i^\dagger(\omega)S_{i'}(\omega')\rho
	\Big) + \text{H.c.}
	\label{eq:Redfield_Dissipator}
\end{equation}


\paragraph{Matrix elements and Redfield tensor.}
In the eigenbasis of the Hamiltonian $\{|a\rangle\}$, taking $\langle a|\,\cdot\,|b\rangle$ of the master equation and grouping terms gives
\begin{equation}
\frac{d}{dt}\rho_{ab}(t)
= -i\,\omega_{ab}\rho_{ab}(t) + \sum_{c,d} R_{abcd}\,\rho_{cd}(t),
\label{eq:bloch_redfield_basis}
\end{equation}
where we defined the Bloch–Redfield tensor that is implemented also in QuTiP as
\begin{align}
R_{abcd}
= -\frac{1}{2\hbar^{2}} \sum_{i, i'}\Big\{&
\delta_{bd}\sum_n S^{i}_{an}S^{i'}_{nc}\,\tilde{S}_{ii'}(\omega_{cn})
- S^{i'}_{ac}S^{i}_{db}\,\tilde{S}_{ii'}(\omega_{ca})
\notag\\
&+\delta_{ac}\sum_n S^{i}_{dn}S^{i'}_{nb}\,\tilde{S}_{ii'}(\omega_{dn})
- S^{i'}_{ac}S^{i}_{db}\,\tilde{S}_{ii'}(\omega_{db})
\Big\}.
\label{eq:Rabcd_full}
\end{align}


\paragraph{Secular approximation.}

\noindent
The oscillatory prefactors $e^{-i(\omega - \omega') t}$ in Eq.~\eqref{eq:Redfield_Dissipator} average to zero on coarse-grained times $\Delta t$ satisfying $\tau_E \ll \Delta t \ll 1/|\omega - \omega'|$ whenever $\omega \neq \omega'$. 
Keeping only terms with $\omega_{ab}=\omega_{cd}$ in Eq.~\eqref{eq:Rabcd_full} removes such fast-rotating couplings and ensures a block-diagonal structure in the Redfield tensor $R_{abcd}$.

This yields the \emph{secular} Redfield (Lindblad form) equation:

\begin{align}
	\dot{\rho}_S(t) & = -i [H_S + H_{\text{LS}}, \rho_S(t)]                                                                                                                                     \\
	                & + \sum_{i,i'} \sum_{\omega} \gamma_{ii'}(\omega) \Big( S_j(\omega) \rho_S(t) S_i^{\dagger}(\omega) - \tfrac{1}{2} \{ S_i^{\dagger}(\omega) S_j(\omega), \rho_S(t) \} \Big),
	\label{eq:Secular_Lindblad_Form}
\end{align}

\noindent
where $\gamma_{ii'}(\omega) = 2 \mathrm{Re}\,\chi_{ii'}(\omega)$. This form is guaranteed to preserve complete positivity provided the matrix $[\gamma_{ii'}(\omega)]_{i,i'}$ is positive semidefinite for each frequency $\omega$. Without the secular approximation, Eq.~\eqref{eq:Redfield_Dissipator} need not generate a completely positive dynamical map, explaining the caution required when applying the full Redfield equation.

\noindent
The corresponding tensor structure can be described as 

\begin{equation}
R_{abcd}^{(\sec)} = \Theta_\epsilon(\omega_{ab} - \omega_{cd}) R_{abcd}, \quad \Theta_\epsilon(\Delta) = \begin{cases} 1, & |\Delta| < \epsilon, \\ 0, & \text{otherwise}. \end{cases}
\end{equation}


\paragraph{Common simplification (uncorrelated Hermitian couplings).}

\noindent
If the baths are uncorrelated and $S_i=S_i^\dagger$:

\begin{align}
\tilde{S}_{ii'}(\omega) &= \delta_{ii'} S_i(\omega) \implies \notag \\
R_{abcd} &= -\frac{1}{2\hbar^{2}} \sum_{i} \Big\{
\delta_{bd} \sum_n S^{i}_{an} S^{i}_{nc} S_{i}(\omega_{cn})
- S^{i}_{ac} S^{i}_{db} S_{i}(\omega_{ca}) \notag \\
&\quad + \delta_{ac} \sum_n S^{i}_{dn} S^{i}_{nb} S_{i}(\omega_{dn})
- S^{i}_{ac} S^{i}_{db} S_{i}(\omega_{db})
\Big\}.
\label{eq:Rabcd_uncor}
\end{align}


%------------------------------------------------------------------------------
%	SECTION 3: Environmental Correlation Functions and Spectral Properties  
%------------------------------------------------------------------------------
\section{Environmental Correlation Functions and Spectral Properties}
\label{sec:environmental_correlation_functions}

\noindent
Having derived the Redfield equation, we now focus on a crucial ingredient: the characterization of the environment through its correlation functions and spectral densities, which encode how environmental fluctuations drive relaxation and dephasing~\cite{breuerpetruccione2009theoryopenquantum, weiss2012quantumdissipativesystems}. These objects translate the abstract operator structure in Eq.~\eqref{eq:Redfield_Markov_TimeLocal} into calculable quantities and connect the microscopic bath model to experimentally accessible spectra.

\noindent
The bath correlation functions determine both the strength and characteristic time scales of the system--environment interaction and bridge quantum and classical noise descriptions. In what follows we show how they arise, state their properties, introduce their Fourier (spectral) representation, and explain how emission and absorption processes are simultaneously encoded.


\subsection{Bath Correlation Functions}
\label{subsec:bath_correlation_functions}

\noindent
The environmental correlation functions entering Eq.~\eqref{eq:Redfield_Markov_TimeLocal} are defined (cf. Eq.~\eqref{eq:Environment_Correlation_Function}) by $C_{ii'}(\tau) = \mathrm{Tr}_E[\hat{E}_i(\tau) \hat{E}_j(0) \hat{\rho}_E(0)] = \langle \hat{E}_i(\tau) \hat{E}_j(0) \rangle_E$, where $\hat{\rho}_E(0)$ is usually a thermal state. For a stationary (equilibrium) bath, $C_{ii'}(\tau)$ depends only on the time difference $\tau$; Hermiticity implies $C_{ii'}^*(\tau)=C_{i'i}(-\tau)$; and the function typically decays on the bath correlation time $\tau_E$, a prerequisite for the Markov approximation when $\tau_E$ is much shorter than the relevant system evolution time scales.


\subsection{Spectral Density and Fourier Representation}
\label{subsec:spectral_density_representation}

\noindent
For a thermal bath, the Kubo--Martin--Schwinger (KMS) condition imposes the detailed-balance symmetry

\begin{equation}
	S_{ii'}(-\omega) = e^{-\hbar\omega/(k_{\mathrm{B}} T)} S_{i'i}(\omega),
	\label{eq:kms_spectral_relation}
\end{equation}

\noindent
which ensures that upward and downward transition rates obey Boltzmann ratios.

\subsection{Physical Interpretation: Emission and Absorption Processes}
\label{subsec:physical_emission_absorption}

\noindent
Emission and absorption are both automatically included because $S_{ii'}(\omega)$ contains positive- and negative-frequency components related by Eq.~\eqref{eq:kms_spectral_relation}. Positive frequencies ($\omega>0$) describe system energy loss (emission), while negative frequencies ($\omega<0$) describe system energy gain (absorption). Detailed balance follows immediately: the ratio of absorption to emission contributions at frequency $\omega>0$ is $e^{-\hbar \omega/(k_{\mathrm{B}}T)}$. For a single bosonic bath with scalar spectral density $J(\omega)$ (defined for $\omega>0$) one often writes the symmetrized spectrum

\begin{equation}
	S(\omega) = 2\pi J(|\omega|) \begin{cases} n_{\text{th}}(\omega)+1, & \omega>0, \\ n_{\text{th}}(|\omega|), & \omega<0, \end{cases}
	\label{eq:bose_symmetric_spectrum}
\end{equation}

\noindent
with Bose--Einstein occupation $n_{\text{th}}(\omega) = \big(e^{\hbar\omega/(k_{\mathrm{B}}T)}-1\big)^{-1}$, explicitly exhibiting stimulated plus spontaneous emission $(n_{\text{th}}+1)$ versus absorption $(n_{\text{th}})$~\cite{weiss2012quantumdissipativesystems}. This formulation makes clear how both processes and their thermal weighting emerge from a single function.


%------------------------------------------------------------------------------
% SECTION 4: Harmonic Oscillator Bath and Correlation Functions
%------------------------------------------------------------------------------
\section{Harmonic Oscillator Baths and Explicit Correlation Functions}
\label{sec:harmonic_oscillator_baths}
Also to do these explicit calculations, we will need the following mathematical result.
\paragraph{Geometric series.}
The infinite geometric series

\begin{equation} \label{eq:ho_infinite_geometric_series}
	S = a + ar + ar^2 + ar^3 + \dots = \sum_{n=0}^{\infty} ar^n,
\end{equation}

\noindent
converges to

\begin{equation} \label{eq:ho_geometric_series_sum}
	S = \frac{a}{1-r} \quad \text{for } |r|<1.
\end{equation}

\noindent
Differentiating Eq.~\eqref{eq:ho_geometric_series_sum} with respect to $r$ gives

\begin{equation} \label{eq:ho_derivation_geometric_sum}
	\sum_{n=0}^{\infty} n r^n = \frac{r}{(1-r)^2}, \quad |r|<1.
\end{equation}



\noindent
We now specialize the formal correlation functions introduced above to the paradigmatic and widely applicable case of a bosonic (harmonic oscillator) environment. Rather than opening a separate chapter, we integrate here the full derivation of thermal correlation functions, spectral density representations, and standard phenomenological models (Ohmic, sub-/super-Ohmic, Drude--Lorentz). This material corresponds to the (former) standalone chapter on bath correlation functions and harmonic oscillator environments.


\subsection{Bosonic Environment and Gibbs State}
\label{subsec:bosonic_environment_gibbs}

\noindent
A bosonic bath is modeled as a (large) collection of independent harmonic oscillators in thermal equilibrium:

\begin{equation} \label{eq:ho_gibbs_state}
	\rho = \frac{e^{-\beta H}}{\mathrm{Tr}[e^{-\beta H}]}, \qquad H=\sum_k \hbar \omega_k \Big(b_k^{\dagger} b_k + \tfrac{1}{2}\Big), \qquad E_k = \hbar \omega_k (n_k + \tfrac{1}{2}),
\end{equation}

\noindent
with $n_k = \langle b_k^{\dagger} b_k \rangle$ the occupation number.


\subsection{Single Mode Partition Function and Occupation}
\label{subsec:single_mode}

\noindent
For a single mode ($H=\hbar \omega b^{\dagger} b$) the thermal state reads

\begin{equation} \label{eq:ho_single_mode_density_matrix}
	\rho = \frac{e^{-\beta \hbar \omega b^{\dagger} b}}{Z},
\end{equation}

\noindent
with partition function

\begin{align} \label{eq:ho_partition_function}
	Z & \equiv \mathrm{Tr}[e^{-\beta H}] = \sum_{m=0}^{\infty} e^{-\beta \hbar \omega (m+1/2)} = \frac{e^{-\beta \hbar \omega/2}}{1 - e^{-\beta \hbar \omega}}.
\end{align}

\noindent
The Bose--Einstein average occupation number follows using Eq.~\eqref{eq:ho_derivation_geometric_sum}:

\begin{align} \label{eq:ho_expectation_number_operator}
	n & = \langle b^{\dagger} b \rangle_{\text{th}} = \frac{\sum_{m=0}^{\infty} m e^{-\beta \hbar \omega m}}{\sum_{m=0}^{\infty} e^{-\beta \hbar \omega m}} = \frac{e^{-\beta \hbar \omega}}{1-e^{-\beta \hbar \omega}} = \frac{1}{e^{\beta \hbar \omega}-1}.
\end{align}

\noindent
For many modes the total partition function factorizes:

\begin{equation} \label{eq:ho_generalized_partition_function}
	Z_{\text{bath}} = \prod_k \frac{e^{-\beta \hbar \omega_k /2}}{1 - e^{-\beta \hbar \omega_k}}.
\end{equation}


\subsection{Microscopic Form of the Bath Correlator}
\label{subsec:microscopic_bath_correlator}

\noindent
With linear system--bath coupling to oscillator displacements (Caldeira--Leggett model \cite{hagstrommorrison2011caldeiraleggettmodel}) we write

\begin{equation} \label{eq:ho_bath_operator}
	B = \sum_{n=1}^{\infty} c_n x_n, \qquad x_n = \sqrt{\frac{1}{2 m_n \omega_n}} (b_n + b_n^{\dagger}).
\end{equation}

\noindent
Time evolution gives

\begin{align}
	B(0)    & = \sum_{n} c_n \sqrt{\frac{1}{2 m_n \omega_n}} (b_n + b_n^{\dagger}), \label{eq:ho_bath_operator_t0}                                                   \\
	B(\tau) & = \sum_{n} c_n \sqrt{\frac{1}{2 m_n \omega_n}} \Big(b_n e^{-i \omega_n \tau} + b_n^{\dagger} e^{i \omega_n \tau}\Big). \label{eq:ho_bath_operator_tau}
\end{align}

\noindent
The thermal correlation function (single relevant bath operator case) is

\begin{equation} \label{eq:ho_bath_correlator}
	C(\tau) = \langle B(\tau) B(0) \rangle.
\end{equation}

\noindent
Using $\langle b_n b_m^{\dagger} \rangle = \delta_{nm}(n_n+1)$ and $\langle b_n^{\dagger} b_m \rangle = \delta_{nm} n_n$ with $n_n$ given by Eq.~\eqref{eq:ho_expectation_number_operator}, we obtain

\begin{equation} \label{eq:ho_correlator_result}
	C(\tau) = \sum_{n} \frac{c_n^2}{2 m_n \omega_n} \Big[(n_n+1) e^{-i \omega_n \tau} + n_n e^{i \omega_n \tau}\Big].
\end{equation}


\subsection{Spectral Density Representation}
\label{subsec:ho_spectral_density}

\noindent
Introduce the (discrete) spectral density

\begin{equation} \label{eq:ho_bath_spectral_density}
	J(\omega) = \pi \sum_{n} \frac{c_n^2}{2 m_n \omega_n} \delta(\omega - \omega_n),
\end{equation}

\noindent
so Eq.~\eqref{eq:ho_correlator_result} becomes

\begin{equation} \label{eq:ho_correlator_spectral_density}
	C(\tau) = \int_{0}^{\infty} d\omega \, \frac{J(\omega)}{\pi} \Big[(n(\omega)+1)e^{-i \omega \tau} + n(\omega) e^{i \omega \tau}\Big],
\end{equation}

\noindent
where $n(\omega)= (e^{\beta \hbar \omega}-1)^{-1}$.


\subsection{Continuum Limit}
\label{subsec:continuum_limit}

\noindent
In the macroscopic limit the mode set becomes dense; introducing a density of states $\rho(\omega)$ and form factor $g(\omega)$ yields $J(\omega)=\rho(\omega) g(\omega)^2$. Using symmetry relations one can recast Eq.~\eqref{eq:ho_correlator_spectral_density} in the standard form

\begin{equation} \label{eq:ho_correlator_final}
	C(\tau) = \int_{0}^{\infty} d\omega \, \frac{J(\omega)}{\pi} \left[ \coth\Big(\frac{\beta \hbar \omega}{2}\Big) \cos(\omega \tau) - i \sin(\omega \tau) \right],
\end{equation}

\noindent
which separates purely dissipative (imaginary) and noise (real, symmetrized) parts.


\subsection{Ohmic and Related Spectral Densities}
\label{subsec:ohmic_spectral_density}

\noindent
A widely used phenomenological family is

\begin{equation} \label{eq:ho_ohmic_spectral_density}
	J(\omega) = \gamma \, \frac{\omega^{s}}{\omega_c^{s-1}} e^{-\omega/\omega_c},
\end{equation}

\noindent
with dimensionless coupling $\gamma$, cutoff frequency $\omega_c$ and exponent $s$ distinguishing Ohmic ($s=1$), sub-Ohmic ($s<1$), and super-Ohmic ($s>1$) cases~\cite{weiss2012quantumdissipativesystems, lambertetal2024qutip5quantum}. The low-frequency linear scaling $J(\omega) \propto \omega$ (Ohmic) underpins frequency-independent damping in many coarse-grained models; the high-frequency cutoff enforces convergence of integrals such as Eq.~\eqref{eq:ho_correlator_final}.

\iffalse
	\begin{figure}[t]
		\centering
		\includegraphics[width=\textwidth]{bath_comparison_combined_0.010_100.00_100.000.png}
		\caption{Comparison of representative bath models (Ohmic and Drude--Lorentz) showing spectral densities, associated power spectra, and time-domain correlation functions for coupling strength $\alpha = 0.1$, cutoff $\omega_c = 100$, and temperature $T=100$. Distinct spectral shapes map directly onto different relaxation and dephasing behaviors.}
		\label{fig:bath_comparison}
	\end{figure}
	% also include a temperature dep. picture
	\begin{figure}
		\centering
		\includegraphics[width=\textwidth]{temperature_analysis_ohmic_bath.png}
		\caption{Temperature dependence of the bath correlation function for an Ohmic bath with $\alpha = 1$ and cutoff $\omega_c = 100$. Higher temperatures increase the amplitude and decrease the correlation time, reflecting enhanced thermal fluctuations.}
		\label{fig:bath_temperature_comparison}
	\end{figure}
\fi
